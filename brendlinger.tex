%====================================================================%
%                  MORIOND.TEX                                       %
%====================================================================%

\documentclass{moriond}

\usepackage{lineno}

% I want to have square brackets
\usepackage[square,numbers]{natbib}

\bibliographystyle{unsrt}    
% for BibTeX - sorted numerical labels by order of
% first citation.

% A useful Journal macro
\def\Journal#1#2#3#4{{#1} {\bf #2}, #3 (#4)}

% Some useful journal names
\def\NCA{\em Nuovo Cimento}
\def\NIM{\em Nucl. Instrum. Methods}
\def\NIMA{{\em Nucl. Instrum. Methods} A}
\def\NPB{{\em Nucl. Phys.} B}
\def\PLB{{\em Phys. Lett.}  B}
\def\PRL{\em Phys. Rev. Lett.}
\def\PRD{{\em Phys. Rev.} D}
\def\ZPC{{\em Z. Phys.} C}

% Some other macros used in the sample text
\def\st{\scriptstyle}
\def\sst{\scriptscriptstyle}
\def\mco{\multicolumn}
\def\epp{\epsilon^{\prime}}
\def\vep{\varepsilon}
\def\ra{\rightarrow}
\def\ppg{\pi^+\pi^-\gamma}
\def\vp{{\bf p}}
\def\ko{K^0}
\def\kb{\bar{K^0}}
\def\al{\alpha}
\def\ab{\bar{\alpha}}
\def\be{\begin{equation}}
\def\ee{\end{equation}}
\def\bea{\begin{eqnarray}}
\def\eea{\end{eqnarray}}
\def\CPbar{\hbox{{\rm CP}\hskip-1.80em{/}}}
%temp replacement due to no font
%%%%%%%%%%%%%%%%%%%%%%%%%%%%%%%%%%%%%%%%%%%%%%%%%%
%                                                %
%    BEGINNING OF TEXT                           %
%                                                %
%%%%%%%%%%%%%%%%%%%%%%%%%%%%%%%%%%%%%%%%%%%%%%%%%%

%\newcommand{\Photo}{\includegraphics[height=35mm]{mypicture}}
\newcommand{\Photo}{}

\begin{document}
\linenumbers

\vspace*{4cm}
\title{ATLAS Higgs Physics Results}

\author{ Kurt Brendlinger, on behalf of the ATLAS Collaboration }

\address{~\\DESY, Notkestra\ss e 85,\\ 22607 Hamburg, Germany}

\maketitle\abstracts{
Recent Higgs boson results are presented using up to 140~fb$^-1$ of proton-proton collision data
delivered by the Large Hadron Collider and recorded by the ATLAS Detector.
The cross section of a Higgs produced in association with a vector boson is measured in phase space
regions called simplified template cross sections,
using events with the Higgs decaying to a $b\bar b$ pair.
An additional measurement of this associated Higgs production is performed in the $H{\rightarrow}WW$
decay channel.
Finally, the production of a Higgs in association iwth a $t\bar t$ pair is measured in the Higgs
diphoton decay channel. In addition, the status of searches for di-Higgs production is reviewed.
}

\section{Introduction}

The Large Hadron Collider (LHC) \cite{Evans:2008zzb} has delivered about 155 fb$^{-1}$ of
collisions, at a center-of-mass energy of $\sqrt{s}=13$~TeV, during its Run 2 data taking program
between 2015 and 2018.
With this data, the
ATLAS Detector \cite{PERF-2007-01} is able to probe the nature of the Higgs boson with unprecedented
precision. The following summarizes the most recent measurements of Higgs boson production using up
to 140 fb$^{-1}$ of high-quality data. These measurements are performed in the $b\bar b$,
diphoton ($\gamma\gamma$) and $WW$ (${\rightarrow}\ell\nu\ell'\nu'$) decay channels, and specifically
target Higgs production in association with a vector boson ($VH$) or in assoication with a top
quark pair ($t\bar tH$).

Higgs measurements are performed in the context of the simplified template cross section framework,
which is designed to measure cross sections of production modes as well as kinematic phase space
regions, while reducing model dependence contained inside the measurements.

\section{Measurement of the $VH$ production mode in the $b\bar b$ decay channel}\label{sec:vh_bb}

Hbb and VH observation \cite{HIGG-2018-04}.

Higgs decaying to $b\bar b$ \cite{HIGG-2018-50}

\begin{figure}[!htbp]
\centering
\includegraphics[width=0.425\linewidth]{figures/HIGG-2018-50/fig_01_v1.pdf}
\includegraphics[width=0.565\linewidth]{figures/HIGG-2018-50/fig_03.pdf}
\caption{
  Left: the BDT. Right: The results.
}
\label{fig:vh_bb}
\end{figure}

\section{Measurement of Higgs production in association with a $t\bar t$ pair in the diphoton decay channel}\label{sec:ttH_yy}

ttH diphoton \cite{ATLAS-CONF-2019-004}

\begin{figure}[!htbp]
\centering
\includegraphics[width=0.525\linewidth]{figures/ATLAS-CONF-2019-004/fig_05_v1.pdf}
\includegraphics[width=0.465\linewidth]{figures/ATLAS-CONF-2019-004/fig_08.pdf}
\caption{
  ttH stuff.
}
\label{fig:vh_bb}
\end{figure}

\section{Measurement of the $VH$ production mode in the $WW$ decay channel} \label{sec:vh_ww}

VH $H{\rightarrow}WW$ \cite{HIGG-2017-14}

Using 36.1~fb$^{-1}$.

\begin{figure}[!htbp]
\centering
\includegraphics[width=0.55\linewidth]{figures/HIGG-2017-14/fig_06b.pdf}
\includegraphics[width=0.44\linewidth]{figures/HIGG-2017-14/fig_08.pdf}
\caption{
  WW stuff.
}
\label{fig:vh_bb}
\end{figure}

\section{Limits on the Higgs Self Coupling} \label{sec:hh}

The limits are presented.

\section{Summary}

The Large Hadron Collider (LHC) \cite{Evans:2008zzb} and the ATLAS Detector.

Many of the final state measurements do not yet use 140~fb$^{-1}$.

% ATLAS-required copyright
% Authors should try to find a place that is not too obtrusive for this statement e.g. as footnote
% on the first page or just before the references
Copyright 2019 CERN for the benefit of the ATLAS Collaboration. CC-BY-4.0 license.

\section*{References}

\bibliography{brendlinger}

\end{document}

%%%%%%%%%%%%%%%%%%%%%%
% End of moriond.tex  %
%%%%%%%%%%%%%%%%%%%%%%
