%====================================================================%
%                  MORIOND.TEX                                       %
%====================================================================%

\documentclass{moriond}

\usepackage{lineno}

% I want to have square brackets
\usepackage[square,numbers]{natbib}

\bibliographystyle{unsrt}    
% for BibTeX - sorted numerical labels by order of
% first citation.

% A useful Journal macro
\def\Journal#1#2#3#4{{#1} {\bf #2}, #3 (#4)}

% Some useful journal names
\def\NCA{\em Nuovo Cimento}
\def\NIM{\em Nucl. Instrum. Methods}
\def\NIMA{{\em Nucl. Instrum. Methods} A}
\def\NPB{{\em Nucl. Phys.} B}
\def\PLB{{\em Phys. Lett.}  B}
\def\PRL{\em Phys. Rev. Lett.}
\def\PRD{{\em Phys. Rev.} D}
\def\ZPC{{\em Z. Phys.} C}

% Some other macros used in the sample text
\def\st{\scriptstyle}
\def\sst{\scriptscriptstyle}
\def\mco{\multicolumn}
\def\epp{\epsilon^{\prime}}
\def\vep{\varepsilon}
\def\ra{\rightarrow}
\def\ppg{\pi^+\pi^-\gamma}
\def\vp{{\bf p}}
\def\ko{K^0}
\def\kb{\bar{K^0}}
\def\al{\alpha}
\def\ab{\bar{\alpha}}
\def\be{\begin{equation}}
\def\ee{\end{equation}}
\def\bea{\begin{eqnarray}}
\def\eea{\end{eqnarray}}
\def\CPbar{\hbox{{\rm CP}\hskip-1.80em{/}}}
%temp replacement due to no font
%%%%%%%%%%%%%%%%%%%%%%%%%%%%%%%%%%%%%%%%%%%%%%%%%%
%                                                %
%    BEGINNING OF TEXT                           %
%                                                %
%%%%%%%%%%%%%%%%%%%%%%%%%%%%%%%%%%%%%%%%%%%%%%%%%%

%\newcommand{\Photo}{\includegraphics[height=35mm]{mypicture}}
\newcommand{\Photo}{}

\begin{document}
\linenumbers

\vspace*{4cm}
\title{ATLAS Higgs Physics Results}

\author{ Kurt Brendlinger, on behalf of the ATLAS Collaboration }

\address{~\\DESY, Notkestra\ss e 85,\\ 22607 Hamburg, Germany}

\maketitle\abstracts{
The Large Hadron Collider and the ATLAS Detector.
}

\section{Introduction}

The Large Hadron Collider (LHC) \cite{Evans:2008zzb} and the ATLAS Detector \cite{PERF-2007-01}.

Citations:
The ATLAS Detector \cite{PERF-2007-01}
Higgs decaying to $b\bar b$ \cite{HIGG-2018-50}
VBF $H{\rightarrow}WW$ \cite{HIGG-2017-14}
ttH diphoton \cite{ATLAS-CONF-2019-004}
Hbb and VH observation \cite{HIGG-2018-04}.

The Moriond proceedings are printed from camera-ready manuscripts.
The following guidelines are intended to get a uniform rending of the 
proceedings. Authors with no connection to \LaTeX{} should use this
sample text as a guide for their presentation using their favorite
text editor (see section~\ref{subsec:final})

\subsection{Producing the Hard Copy}\label{subsec:prod}

The hard copy may be printed using the procedure given below.
You should use
two files: \footnote{You can get these files from
our site at \url{http://moriond.in2p3.fr/proceedings.php}.}
\begin{description}
\item[\texttt{moriond.cls}] the style file that provides the higher
level \LaTeX{} commands for the proceedings. Don't change these parameters.
\item[\texttt{moriond.tex}] the main text. You can delete our sample
text and replace it with your own contribution to the volume, however we
recommend keeping an initial version of the file for reference.
\end{description}
The command for (pdf)\LaTeX ing is \texttt{pdflatex moriond}: do this twice to
sort out the cross-referencing.

%If there is an abbreviation
%defined in the new definitions at the top of the file {\em moriond.tex} that
%conflicts with one of your own macros, then
%delete the appropriate command and revert to longhand. Failing that, please
%consult your local texpert to check for other conflicting macros that may
%be unique to your computer system.
{\bf Page numbers should not appear.}

\subsection{Headings and Text and Equations}

Please preserve the style of the
headings, text fonts and line spacing to provide a
uniform style for the proceedings volume.

Equations should be centered and numbered consecutively, as in
Eq.~\ref{eq:murnf}, and the {\em eqnarray} environment may be used to
split equations into several lines, for example in Eq.~\ref{eq:sp},
or to align several equations.
An alternative method is given in Eq.~\ref{eq:spa} for long sets of
equations where only one referencing equation number is wanted.

In \LaTeX, it is simplest to give the equation a label, as in
Eq.~\ref{eq:murnf}
where we have used \verb^\label{eq:murnf}^ to identify the
equation. You can then use the reference \verb^\ref{eq:murnf}^
when citing the equation in the
text which will avoid the need to manually renumber equations due to
later changes. (Look at
the source file for some examples of this.)

The same method can be used for referring to sections and subsections.

\subsection{Tables}

The tables are designed to have a uniform style throughout the proceedings
volume. It doesn't matter how you choose to place the inner
lines of the table, but we would prefer the border lines to be of the style
shown in Table~\ref{tab:exp}.
 The top and bottom horizontal
lines should be single (using \verb^\hline^), and
there should be single vertical lines on the perimeter,
(using \verb^\begin{tabular}{|...|}^).
 For the inner lines of the table, it looks better if they are
kept to a minimum. We've chosen a more complicated example purely as
an illustration of what is possible.

The caption heading for a table should be placed at the top of the table.

\begin{table}[t]
\caption[]{Experimental Data bearing on $\Gamma(K \ra \pi \pi \gamma)$
for the $\ko_S, \ko_L$ and $K^-$ mesons.}
\label{tab:exp}
\vspace{0.4cm}
\begin{center}
\begin{tabular}{|c|c|c|l|}
\hline
& & & \\
&
$\Gamma(\pi^- \pi^0)\; s^{-1}$ &
$\Gamma(\pi^- \pi^0 \gamma)\; s^{-1}$ &
\\ \hline
\mco{2}{|c|}{Process for Decay} & & \\
\cline{1-2}
$K^-$ &
$1.711 \times 10^7$ &
\begin{minipage}{1in}
$2.22 \times 10^4$ \\ (DE $ 1.46 \times 10^3)$
\end{minipage} &
\begin{minipage}{1.5in}
No (IB)-E1 interference seen but data shows excess events relative to IB over
$E^{\ast}_{\gamma} = 80$ to $100MeV$
\end{minipage} \\
& & &  \\ \hline
\end{tabular}
\end{center}
\end{table}


\subsection{Figures}\label{subsec:fig}

If you wish to `embed' an image or photo in the file, you can use
the present template as an example. The command 
\verb^\includegraphics^ can take several options, like
\verb^draft^ (just for testing the positioning of the figure)
or \verb^angle^ to rotate a figure by a given angle.

The caption heading for a figure should be placed below the figure.

\subsection{Limitations on the Placement of Tables,
Equations and Figures}\label{sec:plac}

Very large figures and tables should be placed on a page by themselves. One
can use the instruction \verb^\begin{figure}[p]^ or
\verb^\begin{table}[p]^
to position these, and they will appear on a separate page devoted to
figures and tables. We would recommend making any necessary
adjustments to the layout of the figures and tables
only in the final draft. It is also simplest to sort out line and
page breaks in the last stages.

\subsection{Acknowledgments, Appendices, Footnotes and the Bibliography}

If you wish to have
acknowledgments to funding bodies etc., these may be placed in a separate
section at the end of the text, before the Appendices. This should not
be numbered so use \verb^\section*{Acknowledgments}^.

It's preferable to have no appendices in a brief article, but if more
than one is necessary then simply copy the
\verb^\section*{Appendix}^
heading and type in Appendix A, Appendix B etc. between the brackets.

Footnotes are denoted by a letter superscript
in the text,\footnote{Just like this one.} and references
are denoted by a number superscript.

Bibliography can be generated either manually or through the BibTeX
package (which is recommanded). In this sample we
have used \verb^\bibitem^ to produce the bibliography.
Citations in the text use the labels defined in the bibitem declaration,
for example, the first paper by Jarlskog~ is cited using the command
\verb^\cite{ja}^.

\section{Measurement of Higgs production in association with a $t\bar t$ pair in the diphoton decay channel}\label{subsec:final}



\section{Measurement of the $VH$ production mode in the $WW$ decay channel} \label{sec:vh_ww}

Using 36.1~fb$^{-1}$.

\section*{Summary}

The Large Hadron Collider (LHC) \cite{Evans:2008zzb} and the ATLAS Detector.

% ATLAS-required copyright
% Authors should try to find a place that is not too obtrusive for this statement e.g. as footnote
% on the first page or just before the references
Copyright 2019 CERN for the benefit of the ATLAS Collaboration. CC-BY-4.0 license.

\section*{References}

\bibliography{brendlinger}

\end{document}

%%%%%%%%%%%%%%%%%%%%%%
% End of moriond.tex  %
%%%%%%%%%%%%%%%%%%%%%%
