%====================================================================%
%                  MORIOND.TEX                                       %
%====================================================================%

\documentclass{moriond}

\usepackage{lineno}

% I want to have square brackets
\usepackage[square,numbers]{natbib}

% For subfigures (e.g. (a) and (b))
\usepackage{subfig}

\bibliographystyle{unsrt}    
% for BibTeX - sorted numerical labels by order of
% first citation.

\usepackage{xspace}
\def\wh{\texorpdfstring{\ensuremath{W\kern -0.1em H}\xspace}{WH\xspace}}
\def\wz{\texorpdfstring{\ensuremath{W\kern -0.1em Z}\xspace}{WZ\xspace}}
\def\vh{\texorpdfstring{\ensuremath{V\kern -0.1em H}\xspace}{VH\xspace}}
\def\zh{\ensuremath{ZH}\xspace}
\def\ttbar{\ensuremath{t\bar{t}}\xspace}
\def\tth{\ensuremath{t\bar{t}H}\xspace}

% A useful Journal macro
\def\Journal#1#2#3#4{{#1} {\bf #2}, #3 (#4)}

% Some useful journal names
\def\NCA{\em Nuovo Cimento}
\def\NIM{\em Nucl. Instrum. Methods}
\def\NIMA{{\em Nucl. Instrum. Methods} A}
\def\NPB{{\em Nucl. Phys.} B}
\def\PLB{{\em Phys. Lett.}  B}
\def\PRL{\em Phys. Rev. Lett.}
\def\PRD{{\em Phys. Rev.} D}
\def\ZPC{{\em Z. Phys.} C}

% Some other macros used in the sample text
\def\st{\scriptstyle}
\def\sst{\scriptscriptstyle}
\def\mco{\multicolumn}
\def\epp{\epsilon^{\prime}}
\def\vep{\varepsilon}
\def\ra{\rightarrow}
\def\ppg{\pi^+\pi^-\gamma}
\def\vp{{\bf p}}
\def\ko{K^0}
\def\kb{\bar{K^0}}
\def\al{\alpha}
\def\ab{\bar{\alpha}}
\def\be{\begin{equation}}
\def\ee{\end{equation}}
\def\bea{\begin{eqnarray}}
\def\eea{\end{eqnarray}}
\def\CPbar{\hbox{{\rm CP}\hskip-1.80em{/}}}
%temp replacement due to no font
%%%%%%%%%%%%%%%%%%%%%%%%%%%%%%%%%%%%%%%%%%%%%%%%%%
%                                                %
%    BEGINNING OF TEXT                           %
%                                                %
%%%%%%%%%%%%%%%%%%%%%%%%%%%%%%%%%%%%%%%%%%%%%%%%%%

%\newcommand{\Photo}{\includegraphics[height=35mm]{mypicture}}
\newcommand{\Photo}{}

\begin{document}
\linenumbers

\vspace*{4cm}
\title{ATLAS Higgs Physics Results}

\author{ Kurt Brendlinger, on behalf of the ATLAS Collaboration }

\address{~\\DESY, Notkestra\ss e 85,\\ 22607 Hamburg, Germany}

\maketitle\abstracts{
Recent measurements of Higgs boson properties are presented using up to 139~fb$^{-1}$ of proton-proton collision data
delivered by the Large Hadron Collider and recorded by the ATLAS detector.
Three measurements are discussed:
first, cross sections of a Higgs boson produced in association with a vector boson are measured
using the $H{\rightarrow}b\bar b$ decay channel.
These cross section measurements, performed in dedicated phase space regions, are called simplified template cross sections.
A measurement of the associated Higgs boson production is also performed in the $H{\rightarrow}WW$
decay channel.
Finally, the production of a Higgs boson in association with a $t\bar t$ pair is measured in the
diphoton decay channel. The results presented here are compatible with Standard Model predictions.
}

\section{Introduction}

The Large Hadron Collider (LHC) \cite{Evans:2008zzb} has delivered about 155 fb$^{-1}$ of proton-proton
collisions at a center-of-mass energy of $\sqrt{s}=13$~TeV during its Run 2 data taking program
between 2015 and 2018.
With this data, the
ATLAS detector \cite{PERF-2007-01,ATLAS-TDR-19} is able to probe the nature of the Higgs boson with unprecedented
precision. The following summarizes the most recent measurements of Higgs boson production using up
to 139 fb$^{-1}$ of high-quality data. These measurements are performed in the $b\bar b$,
and $WW$ (${\rightarrow}\ell\nu\ell'\nu'$) and diphoton ($\gamma\gamma$) decay channels, and specifically
target Higgs boson production in association with a vector boson (\vh) or in association with a top
quark pair ($t\bar tH$).

\section{Measurement of the \vh production mode in the $b\bar b$ decay channel}\label{sec:vh_bb}

The observation of Higgs boson production in association with a $W$ or $Z$ boson, reported in
Ref.~\cite{HIGG-2018-04}, was achieved by combined measurements in the $H{\rightarrow}b\bar b$,
$H{\rightarrow}ZZ^*{\rightarrow}4\ell$ and $H{\rightarrow}\gamma\gamma$ decay channels using
80~fb$^{-1}$ of data collected in Run~2.
Following the observation, the \vh production mode is studied in further detail using Higgs bosons
decaying to a pair of $b$-jets, the most sensitive decay channel for this production mode,
using the same analysis strategy and integrated luminosity \cite{Aaboud:2019nan}.

The cross section is measured in bins of the transverse momentum of the associated gauge boson,
$p^{V}_\mathrm{T}$.
The measurement is performed in the context of the simplified template cross section
(STXS) framework \cite{deFlorian:2016spz,Badger:2016bpw},
which is designed to measure cross sections of production modes as well as kinematic phase space
regions, while reducing the model dependence of the measurements.
Advanced machine learning techniques are used in this analysis to isolate the signal and enhance the experimental
sensitivity of the cross section measurements.

The analysis considers events with 2 $b$-tagged jets and 0, 1 or 2 reconstructed leptons consistent
with vector boson signatures. To extract the signal, eight separate boosted decision trees (BDTs)
are trained, one for each signal region considered.
The correlation between $p^{V}_\mathrm{T}$ and the BDT output score is exploited
to extract the cross section in several regions of $p^{V}_\mathrm{T}$.
Figure~\ref{fig:vh_bb_a} compares the BDT output of the background and the inclusive signal,
as well as two different $p^{V}_\mathrm{T}$
STXS regions, in a signal region targeting \wh production with the $W$ boson decaying leptonically.
The clear shape differences between the STXS regions can be used to
extract both cross sections simultaneously in the fit to data.
%% Cross section measurements in the simplified template
%% scheme can be more precise than a simple fiducial cross section, because the
%% improved discrimination against background and between STXS regions.

\begin{figure}[!htbp]
  \centering
  \subfloat[\label{fig:vh_bb_a}] {\includegraphics[width=0.425\textwidth]{figures/HIGG-2018-50/fig_01_v1.pdf}}
  \subfloat[\label{fig:vh_bb_b}] {\includegraphics[width=0.565\textwidth]{figures/HIGG-2018-50/fig_03.pdf}}
  \caption{
    (a) The BDT shape of the signal and background distributions in the 1-lepton, 2-jet
    reconstructed event category targeting \wh production in the $b\bar b$ decay channel. The shape of two STXS signal regions
    featuring different $p^{V}_\mathrm{T}$ is also shown, highlighting the correlation between
    $p^{V}_\mathrm{T}$ and the BDT response.
    (b) The Higgs boson cross sections in five STXS regions, measured in the $H{\rightarrow}b\bar b$ decay channel.
    The cross section is binned in the $p_\mathrm{T}$ of
    the associated gauge boson, and compared to the Standard Model prediction \cite{Aaboud:2019nan}.
  }
  \label{fig:vh_bb}
\end{figure}

Figure~\ref{fig:vh_bb_b} reports the measured cross sections in several fiducial regions
of the STXS framework. The measurements presented here are also used
to constrain parameters in an effective Lagrangian framework, whose
deviation from predictions would indicate interactions beyond the Standard Model (SM).
The results are consistent with the SM prediction.
The
simplified template cross sections measured in this analysis can be readily combined with those measured in other
decay channels, allowing each decay channel to contribute toward reaching the ultimate precision of
the full Higgs boson fiducial phase space.

\section{Measurement of the \vh production mode in the $WW$ decay channel} \label{sec:vh_ww}

The \vh production of Higgs bosons is also studied in the $H{\rightarrow}WW$ decay channel, with all vector bosons
decaying leptonically,
using 36.1~fb$^{-1}$ of data collected in 2015 and 2016 \cite{HIGG-2017-14}.
To gain sensitivity to the \wh process, events with three reconstructed leptons are separated into
a region depleted in contributions from events with $Z$ bosons by requiring that events have no same-flavor, opposite-charge lepton pair.
Two BDTs, one designed to reject
\ttbar events (BDT$_{\ttbar}$) and one for reducing the \wz background (BDT$_{\wz}$), are trained for
this region. The signal region is composed of six bins in the two-dimensional space defined by the two
BDTs: three bins of BDT$_{\ttbar}$, each of which is subdivided into two BDT$_{\wz}$ bins.

A separate signal region targeting \wh production is constructed using the remaining three-lepton
events, and an additional BDT is used to separate signal events from \wz/$W\gamma^*$ and $ZZ^*$
backgrounds. Finally, the \zh selection is optimized by separating the signal region into events with
one or two same-flavor, opposite-charge lepton pairs.

The \wh and \zh cross sections are extracted using a simultaneous fit of the signal regions as well as
control regions enhanced in the main analysis backgrounds.
Figure~\ref{fig:ww_vh_a} depicts the fitted \wh signal and background in the 6 bins of the
$Z$-depleted signal region.
Figure~\ref{fig:ww_vh_b} compares the measured cross section times branching ratio for $ZH$ and \wh production
to the SM prediction. A mild excess is seen in both cases, though the measured values are
consistent with the SM expectation.

\begin{figure}[!htbp]
  \centering
  \subfloat[\label{fig:ww_vh_a}] {\includegraphics[width=0.55\linewidth]{figures/HIGG-2017-14/fig_06b.pdf}}
  \subfloat[\label{fig:ww_vh_b}] {\includegraphics[width=0.44\linewidth]{figures/HIGG-2017-14/fig_08.pdf}}
  \caption{
    Results from the measurement of \vh production in the $H{\rightarrow}WW$ decay channel.
    (a) The data and fitted signal and backgrounds in the $Z$-depleted \wh signal region, composed of 6 bins in a
    two-dimensional space defined by the BDT$_{\wz}$ and BDT$_{\ttbar}$ scores---see the text.
    The shaded band includes statistical and systematic uncertainties on both signal and background as estimated by the fit.
    (b) The best-fit cross section times branching ratio for \wh and \zh production and the 68\% and
    95\% confidence-level contours, compared to the SM prediction \cite{HIGG-2017-14}.
  }
  \label{fig:ww_vh}
\end{figure}

\section{Measurement of Higgs boson production in association with a $t\bar t$ pair in the diphoton decay channel}\label{sec:ttH_yy}

The measurement of Higgs boson production with an associated \ttbar pair (\tth production) and decay
to two photons has been updated to include the full Run~2 dataset \cite{ATLAS-CONF-2019-004}.
In this analysis, reconstructed events are sorted into two regions: one targeting {\itshape hadronic} decays
of the \ttbar pair and one for {\itshape leptonic} decays of at least one of the top quarks. Two BDTs are developed, one
for hadronic and one for leptonic events, which give discrimination against both non-resonant
background and non-\tth Higgs boson production. Seven categories are defined based on the BDT scores
(3 BDT bins in the leptonic events and 4 BDT bins in the hadronic events), and a 
simultaneous fit of the invariant diphoton mass spectra in every category is used to extract the
\tth signal.
%%
%% A simultaneous fit of the invariant diphoton mass spectrum in the 7 BDT categories is performed to
%% measure the \tth cross section. The \tth signal and non-\tth Higgs boson production background are
%% modeled with functions developed from simulated samples, while the continuum background is modeled
%% with a power law function or an exponential function, chosen separately in each category based on their
%% performance in data control regions that closely resemble the signal regions.
Figure~\ref{fig:tth_a} shows the sum of the signal and background fits in all 7 categories, where
the contribution from each channel is weighted by an approximate measure of its expected significance.
In Figure~\ref{fig:tth_b}, the contribution of \tth signal, other Higgs boson
production processes, and continuum background is shown for each category.

\begin{figure}[!htbp]
  \centering
  \subfloat[\label{fig:tth_a}] {\includegraphics[width=0.525\linewidth]{figures/ATLAS-CONF-2019-004/fig_05_v1.pdf}}
  \subfloat[\label{fig:tth_b}] {\includegraphics[width=0.465\linewidth]{figures/ATLAS-CONF-2019-004/fig_08.pdf}}
  \caption{
    Results from the measurement of \tth production in the diphoton decay channel.
    (a) The invariant mass spectrum of the diphoton system for the sum of all signal regions, weighted
    by the approximate expected significance of each region. The fitted total background includes non-resonant
    backgrounds as well as other (non-\tth) Higgs boson production modes.
    (b) The number of data events in the diphoton invariant mass window containing 90\% of the expected
    \tth signal, in each BDT bin of the hadronic and leptonic categories.
    The fitted \tth signal, continuum background, and non-\tth Higgs boson production are shown in each bin \cite{ATLAS-CONF-2019-004}.
  }
  \label{fig:tth}
\end{figure}

The measured \tth production cross section times branching ratio
is:
%
\begin{equation}
\sigma_{\tth} \times B_{\gamma\gamma} = 1.59~^{+0.38}_{-0.36} ~\textrm{(stat.)}
~^{+0.15}_{-0.12} ~\textrm{(exp.)}
~^{+0.15}_{-0.11} ~\textrm{(th.)}~\textrm{fb},
\end{equation}
%
which is 1.4 times the SM prediction.
The measurement has an observed significance of 4.9$\sigma$ (4.2$\sigma$ expected),
highlighting the importance of the diphoton decay channel in the study of \tth production.

%% \section{Limits on the Higgs boson Self Coupling} \label{sec:hh}

%% The limits are presented.

\section{Summary}

The LHC has delivered an unprecedented amount of proton-proton collision data in Run~2, which ATLAS has begun to
fully exploit. The results presented here, using up to 139~fb$^{-1}$ of collision data,
represent a continuing effort to fully characterize Higgs boson production,
as we move into to the precision era of Higgs physics.

%% The \tth result, along with the measurements of $VH$
%% production presented here, are part of the continued effort to fully characterize the properties
%% of the Higgs boson as we move into the precision era of Higgs physics.
% ATLAS-required copyright
% Authors should try to find a place that is not too obtrusive for this statement e.g. as footnote
% on the first page or just before the references
\phantom{.}

\noindent
Copyright 2019 CERN for the benefit of the ATLAS Collaboration. CC-BY-4.0 license.

\begin{thebibliography}{99}
\bibitem{Evans:2008zzb} L. Evans and P. Bryant, {\em JINST}, 3:S08001, 2008.
\bibitem{PERF-2007-01} ATLAS Collaboration, {\em JINST}, 3:S08003, 2008.
\bibitem{ATLAS-TDR-19} ATLAS Collaboration, ATLAS-TDR-19, http://cds.cern.ch/record/1291633, 2010.
\bibitem{deFlorian:2016spz} D. de Florian et al., CERN-2017-002-M, http://cds.cern.ch/record/2227475, 2017.
%\bibitem{Badger:2016bpw} J. R. Andersen et al., FERMILAB-CONF-16-175-PPD-T,\\ http://cds.cern.ch/record/2153502.
\bibitem{Badger:2016bpw} J. R. Andersen et al., \href{http://arxiv.org/abs/1605.04692}{\color{black}{arXiv:1605.04692}}, 2016.
\bibitem{HIGG-2018-04} ATLAS Collaboration, {\em Phys. Lett.} B786:59, 2018.
\bibitem{Aaboud:2019nan} ATLAS Collaboration, {\em JHEP}, 05:141, 2019.
\bibitem{HIGG-2017-14} ATLAS Collaboration, CERN-EP-2019-038, submitted to {\em Phys. Lett.} B, 2019.
\bibitem{ATLAS-CONF-2019-004} ATLAS Collaboration, ATLAS-CONF-2019-004, http://cds.cern.ch/record/2668103,\\ 2019.
% \bibitem{Aaboud:2018urx} ATLAS Collaboration, {\em Phys. Lett.} B784:173-191, 2018.
\end{thebibliography}

% \bibliography{brendlinger}

\end{document}

%%%%%%%%%%%%%%%%%%%%%%
% End of moriond.tex  %
%%%%%%%%%%%%%%%%%%%%%%

% LocalWords:  BDT STXS distinguishability
